\section {Das Schreiben mit \LaTeX}
\label{a:latex}

\subsection{Das Besondere an \LaTeX}

In diesem Kapitel werden Sie einige nützliche Hinweise zur
Erstellung von Bachelor- oder Masterarbeiten mit \LaTeX\ erhalten.
Wichtig zu wissen ist, dass \LaTeX\ kein WYSIWYG (what you see is
what you get) Programm ist wie andere Texteditoren wie z.B.
Microsoft Word. Es ist vielmehr eine Art Programmiersprache, mit
der Sie den logischen Zusammenhang Ihres Textes durch geeignete
Befehle zunächst beschreiben. Dieser ''Quellcode'' ist die sog.\
\LaTeX-Datei.

Es gibt wie bei Programmen eine compilierbare Datei für das
Hauptprogramm und u.U.\ weitere Dateien für Unterprogramme, die
eingebunden werden (sinnvollerweise sind das hier die Kapitel).
Der Compiler erzeugt aus dem Quellcode entweder eine sog.\
dvi-Datei (für ''device independent'') (\texttt{LaTeX}) oder direkt eine
pdf-Datei (\texttt{pdfLaTeX}). Aus der dvi-Datei können Sie mit dem
Programm \texttt{dvi2ps} dann eine Postscript-Datei erstellen, die dann auf
geeigneten Druckern ausgedruckt werden kann.

Dies gibt Ihnen z.B.\ auch die Möglichkeit im Quellcode Kommentare
einzufügen, die im endgültigen Text nicht sichtbar sind. Es kommt
dabei auch nicht auf die Zeilenumbrüche im Quellcode an, da diese
erst durch den Compilierungsvorgang erzeugt werden. Das Layout der
fertigen Arbeit können Sie getrost in der letzten Woche Ihrer Arbeit
machen; wenn alle logischen Verknüpfungen und der Text als solcher
sowie die Bilder stimmig sind, dauert das ca.\ einen bis zwei Tage. 
Bedenken Sie, dass auch das Ausdrucken, ins Besondere bei farbigen 
Seiten Zeit in Anspruch nimmt!

Die Vorteile von \LaTeX\ werden ins Besondere bei Umstellungen des
Textes und beim Formelsatz deutlich. Durch die logische
Referenzierung stimmen die Referenzen immer, die Druckqualität von
Formeln wird von anderen Programmen nicht erreicht. Die
Einfachheit des Zitierens ist gerade für einen Anfänger sehr
hilfreich.

Der folgende Text als solcher ist nicht sonderlich sinnvoll, wenn
Sie aber gleichzeitig die Datei mit dem Quellcode lesen, kann man
sehr gut verstehen, wie verschiedene Elemente erzeugt werden. Sie
sollten dann diese Datei auch selbst compilieren und das Ergebnis
vergleichen. Wenn es nicht identisch ist, ist u.U.\ die
\LaTeX-Konfiguration nicht korrekt und muss dann korrigiert werden.

\subsection{Normaler Text}

Normalen Text zu schreiben ist ganz einfach. Man nutzt einen
beliebigen Editor und schreibt einfach den Text, ohne auf die
Formatierung, Zeilenumbrüche etc. zu achten. Will man einen neuen
Abschnitt beginnen, so lässt man einfach eine oder mehrere
Leerzeilen...

Hier sind wir nun im neuen Abschnitt. Die Schriftgröße wird
übrigens im Kopfteil des Hauptdokumentes definiert. Man kann
Hervorhebungen durch andere Schrifttypen machen. Normalerweise
werden neu definierte Begriffe \textit{kursiv}, wichtige
Aussagen \textbf{fett} und Programmzeilen im \texttt{Schreibmaschinenstil}
gesetzt.

\subsubsection{Gliederungsebenen}

Man kann feinere Unterteilungen der Abschnitte durch den Befehl
\texttt{subsubsection} erreichen. Diese Ebene sollte die unterste Ihrer
Arbeit sein.

\subsubsection*{Gleiche Ebene, aber nicht numeriert}

Sehen Sie in den Quellcode, um zu wissen, wie das erreicht wird!

\subsection{Besondere Objekte}


\subsubsection{Abbildungen}\label{b:picturesubsection}

Abbildung~\ref{f:picfirstfigure} zeigt ein Blockschaltbild.

\begin{figure}[ht]		% h - here, t - top, b - bottom, p - page, ! - try hard
  \centering
  \afig{1}{figures/example}			% {scaling}{Figure from MATLAB, picture, etc.}
  \caption{Das erste Bild}
  \label{f:picfirstfigure}
\end{figure}
 
Weitere kostenlose Vektorgrafikprogramme sind
\begin{itemize}
	\item Inkscape,
	\item \LaTeX Draw,
	\item \textsc{Matlab} mit Matlab2Tikz
	\item ...
\end{itemize}

Viele Bilder werden üblicherweise direkt unter \textsc{Matlab} erzeugt, bitte beachten Sie die
Export-Optionen, die der Befehl ''print'' bietet. Sie können aber auch die Möglichkeiten von \textsc{Matlab2Tikz} nutzen. Siehe auch \href{https://github.com/matlab2tikz/matlab2tikz}{Github:Matlab2Tikz}. Sie können sowohl Blockdiagramme

\begin{figure}[!ht]
    \centering
    \tikzfig{1}{closedloop_L}
    \caption{Dies ist ein Blockdiagramm}
    \label{fig:c1_closedloop_L}
\end{figure}

als auch Figures aus \textsc{Matlab}, die Sie mit den Befehlen in eine tex file umwandeln können.
Sie solten aus Gründen der Geschwindigkeit und der Kompilierbarkeit auch den Befehl \texttt{cleanfigure} ausführen, um die Daten an die Auflösung zum Drucken anzupassen. Dieses reduiziert die Dateigröße drastisch.

\begin{lstlisting}
cleanfigure('targetResolution', 300);
matlab2tikz(filename, 'width', '12cm', 'height','6cm');
\end{lstlisting}

\begin{figure}[!ht]
    \centering
    \setlength{\figw}{0.8\textwidth}
    \setlength{\figh}{0.4\textwidth}
    \tikzfig{1}{figure2}
    \caption{Die ist eine Figure aus \textsc{Matlab} umgewandelt mit \textsc{Matlab2Tikz}}
    \label{fig:c1_figure2}
\end{figure}


\subsubsection{Formeln}

Formeln wie diese
\begin{eqnarray}
\ddot{\phi}_1
    &=&
    \frac{M_1+l_1 \sin \phi_2
    (m_2 l_{s2} + m_3 l_{s3})
    (\dot{\phi}_2^2+2 \dot{\phi}_1\dot{\phi}_2 )
    -f_1 \dot{\phi}_1}
    {\theta_1+\theta_2+\theta_3+2l_1 (m_1 l_{s2}+m_3 l_{s3} \cos \phi_2)+
    m_3 (l_1^2+l_2^2)+m_l l_1^2}  \,,  \label{e:eqnfirst} \\
\ddot{\phi}_2
    &=&
    \frac{M_2 + l_1 \sin \phi_2 (m_2 l_{s2} + m_3 l_{s3})
    \dot{\phi}_1^2 +2 - \phi_2 \dot{\phi}_2}
    {\theta_1+\theta_2+\theta_3} \, \label{e:eqnsecond}
\end{eqnarray}
können in schönem Satz ausgedruckt werden. Bitte beachten Sie,
dass auch Formeln zu den Sätzen gehören und ebenso Satzzeichen
enthalten können!


\subsubsection{Symbole}

Symbole wie $\Omega$ können auch einfach in den Fliesstext mit
aufgenommen werden. Sätze fangen {\bf nie} mit einem Symbol an!


\subsubsection{Zitate}

Zitate werden einfach mit Komma an den Satz angehängt,
\cite{We14}. Nachdem man das Programm BiB\TeX\ aufgerufen hat, wird
das Literaturverzeichnis automatisch erstellt. Die Auswahl eines
Zitierstiles erfolgt in der Haupt-Datei.


\subsubsection{Inhaltsverzeichnis}

Das Inhaltsverzeichnis wird automatisch durch \LaTeX\ erstellt. Dazu
schreibt das Programm bei jedem Compilierungslauf sog.\
\texttt{aux}-Dateien (auxiliary), die alle für das Inhaltsverzeichnis
wichtigen Elemente enthalten. Diese Dateien werden dann beim
nächsten Compilieren mit eingebunden. Das gleiche passiert mit allen
Referenzen auf Abbildungen, Formeln, etc. Eine leicht vereinfachte, aber
intuitive Erklärung besteht darin, dass sich \LaTeX\ gewissermaßen Notizen
macht während es Ihren Text von oben nach unten durchließt. Beim zweiten
Durchgang enthält der Notizblock die Information, die zum Schließen der
Verbindungen benötigt werden. Fast so, wie wenn Sie ein Vorlesungsskript studieren...


\subsubsection{Referenzen}\label{b:subsecrefer}

Man kann auf Abbildungen wie die Abbildung~\ref{f:picfirstfigure}, 
Gleichungen~(\ref{e:eqnsecond}), oder ganze Abschnitte~\ref{b:picturesubsection}
verweisen. Sehen Sie, wie das geht? Hierin liegt die eigentliche
Stärke des Programms!

Übrigens sind die Beschriftungen unter den Abbildungen nicht wie Titel kapitalisiert, weil
es sich um Beschreibungen und nicht um Titel handelt.


\subsubsection{Tables}

Tabellen zu erzeugen ist so ziemlich das aufwändigste, was man beim Schreiben einer Arbeit 
in \LaTeX\ tun muss---und das sagt etwas über die Eleganz und Einfachheit von \LaTeX \ aus!

Bei genauem Hinschauen erkennt man, dass Tabelle~\ref{t:table} keine vertikalen und horizontale Trennstriche unterschiedlicher Linienstärken verwendet.
So sollten Tabellen aussehen. Fühlt man sich in der Verlegenheit, vertikale Trennstriche einzusetzen, ist vermutlich das gesamte
Tabellendesign nicht sonderlich gut leserlich.

\begin{table}
	\caption{Eine wunderschöne Tabelle}
	\label{t:table}
	\begin{center}
	\begin{tabular}{ccccc}
	\toprule
                          	&        & \multicolumn{3}{c}{Spalten} \\
    \cmidrule{3-5}
                            & No.    & A     & B     & C           \\
    \midrule
	\multirow{3}{*}{Zeilen} & 1      & A1    & B1    & C1          \\
	                        & 2      & A2    & B2    & C2          \\
	                        & 3      & A3    & B3    & C3          \\
	\bottomrule
	\end{tabular}
	\end{center}
\end{table}

Hier kann man sich über grundlegende Regeln für guten Tabellensatz informieren:

\href{http://tug.org/pracjourn/2007-1/mori/mori.pdf}{``\textit{Tables in \LaTeX\ 2${}_\varepsilon$: Packages and Methods}''}.
