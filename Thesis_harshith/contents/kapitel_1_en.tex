\section{INTRODUCTION}\label{a:latex_en}

\subsection{Background}
In the model-based approach, having a mathematical representation of the plant is crucial for devising a specific controller. Modeling the plant is a pivotal yet challenging aspect of these methods. As outlined in [\cite{panizza2015}], model identification serves as a means to obtain the plant model by utilizing data from experimental tests conducted on the actual system. Various identification techniques can be employed for this purpose: within the black-box framework, the model is directly derived from input-output data alone, while grey-box algorithms first establish a physically grounded model based on fundamental principles and then adjust the model parameters using experimental data. Nonetheless, regardless of the sophistication of the identification method employed, the model always serves as an approximation of the actual system, leading to inevitable errors. In addition, it's imperative to acknowledge the diverse landscape of machine learning models applied in control systems\cite{DEV20213744}. This research specifically delves into Gaussian Process (GP) model controls, amidst the plethora of techniques utilized within the field. By focusing on GP models, the study aims to explore their efficacy within the realm of black box modeling for nonlinear systems, particularly in the construction of Model Predictive Controllers (MPCs) to regulate system dynamics.

\subsection{Motivation}
Despite many recent advancements \cite{silver2016mastering} \cite{yahya2016collective}, data-driven approaches face a significant drawback: their learning process is slow, requiring an impractical number of interactions with the environment to accurately model the system. This limitation poses a practical challenge in real-world systems like robots, where numerous interactions can be time-consuming and impractical. Such data inefficiency renders learning in control and robotic systems impractical and hinders the application of data-driven approaches in more complex scenarios. To enhance data efficiency, either task-specific prior knowledge or the extraction of more information from available data is required. However, obtaining task-specific prior knowledge is often impossible, especially in cases such as robotics. Instead, a viable alternative involves modeling observed dynamics using a flexible nonparametric approach. Generally, model-based methods, which involve learning an explicit dynamics model of the environment, hold more promise for efficiently extracting valuable information from available data compared to model-free methods. Despite their potential, model-based methods are not widely used due to their susceptibility to model errors. These methods inherently assume that the learned model accurately resembles the real environment, which can severely impact their effectiveness.


\subsection{Approach}
A promising approach to enhance the data efficiency of the Model-Based Learning Model without relying on task-specific prior knowledge is to learn models of the underlying system dynamics. When a high-quality model is available, it can effectively serve as a stand-in for the real environment. This means that beneficial policies can be derived from the model without the need for additional interactions with the actual system. However, accurately modeling the underlying transition dynamics presents a significant challenge and inevitably introduces model errors \cite{schneider1997exploiting}.

To address these model errors, probabilistic models have been proposed specifically the Probabilistic Gaussian process \cite{deisenroth2011pilco}. These models explicitly incorporate uncertainties about the system dynamics. By considering model uncertainty, Model-based learning algorithms can substantially reduce the number of interactions required with the real system. This approach acknowledges that while perfect models are often unattainable, probabilistic models offer a more robust framework for navigating the inherent uncertainties in the modeling process. Thus, leveraging probabilistic models can lead to more efficient learning and improved performance in learning tasks by better accommodating the inevitable inconsistencies between the learned model and the real system\cite{kamthe2018}.


This probabilistic Gaussian process has a limitation in that it cannot accept Gaussian input and predict over the entire prediction horizon (detailed explanation in Section \ref{sec:section2.3}). To address this, a transition model is proposed for long-term prediction, which serves as input to the MPC controller. However, an open-loop controller cannot stabilize the system on its own. Therefore, obtaining a feedback controller is essential. Model Predictive Control (MPC) provides a practical framework for this purpose \cite{mayne2000constrained}. During interaction with the system, MPC determines an $N$-step open-loop control trajectory $u_0, \ldots, u_{N-1}$, starting from the current state $x_t$. Only the first control signal $u_0$ is applied to the system\cite{Berberich_2021}. When the system transitions to $x_{t+1}$, the Gaussian Process (GP) model is updated with the newly available information, and MPC re-plans $u_0, \ldots, u_{N-1}$. This process effectively transforms an open-loop controller into an implicit closed-loop (feedback) controller by continuously re-planning $N$ steps ahead from the current state, as illustrated in Figure \ref{f:figure1}.

% This Probabilistic Gaussian process has a limitation, that is it cannot take gaussian input and predict over the entire prediction horizon (detailed explanation in Section \ref{sec:section2.3}), a transition model is proposed for long-term prediction. This transition model serves as input to the MPC controller. However, an open-loop controller cannot stabilize the system. Therefore, it is essential to obtain a feedback controller. Model Predictive Control (MPC) is a practical framework for the model \cite{mayne2000constrained}. While interacting with the system, MPC determines an $N$-step open-loop control trajectory $u_0, \ldots, u_{N-1}$, starting from the current state $x_t$. Only the first control signal $u_0$ is applied to the system. When the system transitions to $x_{t+1}$, we update the Gaussian Process (GP) model with the newly available information, and MPC re-plans $u_0, \ldots, u_{N-1}$. This procedure turns an open-loop controller into an implicit closed-loop (feedback) controller by repeatedly re-planning $N$ steps ahead from the current state as shown in Figure \ref{f:figure1}.

 \begin{figure}[ht]		% h - here, t - top, b - bottom, p - page, ! - try hard
  \centering
  \afig{0.7}{figures/blockdia}			% {scaling}{Figure from MATLAB, picture, etc.}
  \caption{Block Diagram of GP-MPC Controller Framework}
  \label{f:figure1}
\end{figure}



\subsection{Outline}
The thesis starts with the introduction of the utilized model, in Section \ref{sec: GP-2}, which mainly focuses on Gaussian process regression in supervised learning, and after that the literature concerning the application of Gaussian process regression in control and focus on the learning of dynamic models. Furthermore, necessary assumptions to guarantee the applicability of GPs are described. In addition, the GP transition model is introduced for long-term prediction with a detailed explanation. In Section \ref{sec: MPC_3}, focuses on Model Predictive control and its basics. In addition, a detailed framework of the GP-MPC learning controller. Section \ref{sec:results_rts_en} discusses the implementation and evaluation of the proposed approach through simulations on both linear and nonlinear systems. Finally, Section \ref{sec: Conc} provides a summary of the findings and results. It outlines directions for future research in the field of data-efficient control methodologies.


% % This section gives some useful hints to write a thesis with \LaTeX. It is important to know that \LaTeX\ is not a WYSIWYG (what you see is what you get) program like other text editors, such as Microsoft Word.
% Instead it much more resembles a programming language, in which you construct your text by proper usage of syntax. The ''source code'' is your \LaTeX file (\texttt{.tex}). 

% As in other programming languages it is possible to insert comments in \LaTeX\ that are not visible in the final text. 
% Line wraps are not of importance while writing the text, since they are created during compilation. Therefore, formatting 
% with \LaTeX is not a big deal and should not take a lot of time, if all the logical relations are correct.

% One of the big advantages of \LaTeX\ is writing formulas. Using the logical referencing of your formulas, those references are 
% always correct, even if you change the position of the formulas. Furthermore the print quality you achieve with \LaTeX\ formulas 
% is hardly matched by any other program, let alone free of charge!
% Citing is very easy in \LaTeX, as well.


% The following text is not very meaningful on its own, but if you read the source code at the same time, it is easy to understand how 
% different elements are constructed. You should try to compile the file yourself and compare the results. If there are any differences,
% check if your \LaTeX-configurations are correct.


% \subsection{Motivation}

% Writing continuous text is very easy. You can use an arbitrary text editor and just start writing, without paying any attention to 
% formatting, line breaks, etc. If you want to start a new paragraph, just leave one or more blank lines...

% So here we are now in a new paragraph. The font size is defined in the header of your main file. Emphasis is possible by different font styles. 
% Thus when you define a new term, this is normally accentuated in \textit{italic}, important statements in \textbf{bold} and programming code in
% \texttt{typewriter} or \verb"verbatim" style.


% \subsection{Main Objective}


% \subsubsection{Figures}\label{b:picturesubsection}

% Figure~\ref{f:picfirstfigure} shows a block diagram.

%  \begin{figure}[ht]		% h - here, t - top, b - bottom, p - page, ! - try hard
%   \centering
%   \afig{1}{figures/example}			% {scaling}{Figure from MATLAB, picture, etc.}
%   \caption{First figure}
%   \label{f:picfirstfigure}
% \end{figure}

% \begin{figure} [ht]
%     \centering
%     \includegraphics[width=0.5\linewidth]{figures/fig1bd.png}
%     \caption{Block diagram}
%     \label{fig:enter-label}
% \end{figure}
% If you compile your file using \texttt{dvi} 
% (\texttt{latex} $\rightarrow$ \texttt{dvi} $\rightarrow$ \texttt{pdf}, oder \texttt{latex} $\rightarrow$ \texttt{dvi} $\rightarrow$ \texttt{ps} $\rightarrow$ \texttt{pdf}) 
% your Figure~\ref{f:picfirstfigure} can be either a PS or an EPS file. If you are compiling with \texttt{pdfTeX}
% (\texttt{latex} $\rightarrow$ \texttt{pdf}), the figures need to be stored as JPEG, PDF, PNG, etc.\ files---not in PS or EPS format.

% Figure~\ref{f:xfigfigure} is an example for a diagram, created with XFig, a free and open source vector graphics editor.
% Other notable vector graphics editors are
% \begin{itemize}
% 	\item Inkscape,
% 	\item \LaTeX Draw,
% 	\item ...
% \end{itemize}

% \begin{figure}[ht]
%   \centering
%   % \xfig{1}{carts.fig}			% {scaling}{XFig figure}
%   \caption{The second figure}
%   \label{f:xfigfigure}
% \end{figure}

% \subsubsection{Formulas}

% Formulas like
% \begin{eqnarray}
% \ddot{\phi}_1
%     &=&
%     \frac{M_1+l_1 \sin \phi_2
%     (m_2 l_{s2} + m_3 l_{s3})
%     (\dot{\phi}_2^2+2 \dot{\phi}_1\dot{\phi}_2 )
%     -f_1 \dot{\phi}_1}
%     {\theta_1+\theta_2+\theta_3+2l_1 (m_1 l_{s2}+m_3 l_{s3} \cos \phi_2)+
%     m_3 (l_1^2+l_2^2)+m_l l_1^2}  \,,  \label{e:eqnfirst} \\
% \ddot{\phi}_2
%     &=&
%     \frac{M_2 + l_1 \sin \phi_2 (m_2 l_{s2} + m_3 l_{s3})
%     \dot{\phi}_1^2 +2 - \phi_2 \dot{\phi}_2}
%     {\theta_1+\theta_2+\theta_3} \, \label{e:eqnsecond}
% \end{eqnarray}
% are typeset nicely. Keep in mind that formulas are and should be written as part of sentences and thus can and have to contain punctuation marks!


% \subsubsection{Symbols}

% Symbols like $\Omega$ can be included inline with the text. \textbf{Never} start a sentence with a symbol!


% \subsubsection{Citations}

% To cite a reference use the command \verb'\cite{We14}' as done here: \cite{We14}. Here \texttt{We14} is the so called BiB\TeX-Key. Have a look into the file ``S:/Standards/BibTex/rts''
% to see what that means. References usually are simply attached at the end of a statement, separated by a comma,
% \cite{We14}. Be aware that \LaTeX\ in general provides the command \verb'\cite'.

% \subsubsection{Index of Contents}

% The table of contents is generated automatically by \LaTeX. Therefore, each time you compile your main file, an \texttt{aux}-file (auxiliary) is generated that contains all 
% information for the table of contents. This file is embedded, when you compile a second time. This holds true for all references and any change you make with respect to these
% requires to compile twice. An intuitive and simplified explanation is that \LaTeX\ just ''reads'' your files from top to bottom, taking notes in the \texttt{aux}-file while doing so.
% When reading your files again, the notes tell \LaTeX\ that it came across some figure, table or equation before, just like you when studying a subject.


% \subsubsection{References}\label{b:subsecrefer}

% It is possible to refer to figures, equations, etc. like this: Figure~\ref{f:picfirstfigure}, Equation~(\ref{e:eqnsecond}),
% Sections~\ref{b:picturesubsection}. See how easy that is. This is one of the main advantages of \LaTeX!
% Note that ''Equation~(\ref{e:eqnsecond})'', e.g., is capitalized. It is regarded as the figures own name. Would you
% write your own name in lower case?

% While we are at it, have a look at the figure's caption. It uses capitalization like any other sentence in this document.
% Why? Because it is \textbf{not} a title, but a description.

% \subsubsection{Tables}

% Writing tables is probably the most cumbersome writing stuff in \LaTeX\ can get---and that is saying quite something!
% Have a close look at Table~\ref{t:table}. It does not use vertical rules and different line weights in respective places.
% This is the way to design tables. If you feel the need to use vertical rules, your table is probably not very readable in
% the first place.

% \begin{table}
% 	\caption{A beautiful table}
% 	\label{t:table}
% 	\begin{center}
% 	\begin{tabular}{ccccc}
% 	\toprule
%                           	&        & \multicolumn{3}{c}{Columns} \\
%     \cmidrule{3-5}
%                             & Items  & A     & B     & C           \\
%     \midrule
% 	\multirow{3}{*}{Rows}   & 1      & A1    & B1    & C1          \\
% 	                        & 2      & A2    & B2    & C2          \\
% 	                        & 3      & A3    & B3    & C3          \\
% 	\bottomrule
% 	\end{tabular}
% 	\end{center}
% \end{table}

% Inform yourself about some basic rules on typesetting tables:

% \href{http://tug.org/pracjourn/2007-1/mori/mori.pdf}{``\textit{Tables in \LaTeX\ 2${}_\varepsilon$: Packages and Methods}''}.
