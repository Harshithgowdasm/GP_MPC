\section{Conclusion and Outlook} \label{sec: Conc}

\subsection{Conclusion}
In this thesis, we introduced a probabilistic Model Predictive Control (MPC) approach utilizing Gaussian process models to handle the challenges posed by noisy and uncertain system dynamics. Through comprehensive simulations on both linear (DC motor) and nonlinear (Van der Pol oscillator) systems, we validated the efficacy of the proposed GP-MPC framework under various conditions.

In the case of the DC motor, the validation process of the Gaussian process (GP) model and long-term prediction model showcased their reliability and accuracy in capturing system dynamics, even in the presence of noise and uncertainty. By achieving small root mean square error (RMSE) values and average variances, these models demonstrated their capability to make trustworthy predictions in real-world scenarios. When applied to the noise-free DC motor system, GP-MPC exhibited precise control performance akin to direct MPC, highlighting the effectiveness of deterministic modeling in control design. Furthermore, when subjected to uncertainty and noisy measurements, the GP-MPC controller adapted quickly and showcased robust performance comparable to the noise-free scenario. This adaptability was attributed to the GP model's ability to incorporate observed state transitions, enabling the controller to learn and adapt to uncertainty over time.

Similarly, the comprehensive examination of the Van der Pol oscillator within the context of Gaussian Process Model Predictive Control (GP-MPC) has provided valuable insights into its efficacy across various scenarios, including both deterministic and uncertain environments. The validation of the GP model and its derivative long-term prediction model underscores their reliability and accuracy in capturing system dynamics, even in the presence of nonlinearity and noisy measurements. While deviations between predicted and actual values may occur, particularly in uncertain scenarios, the overall performance of the GP-MPC framework remains commendable, because of the small prediction horizon which has fewer errors and satisfactory accuracy for modeling the system dynamics.  

In both scenarios without uncertainty and with uncertainty, the GP-MPC framework struggles to control at the beginning of the horizon but overall it demonstrates faster control performance and achieves desired setpoints within reasonable settling times(8-14 timesteps). However, in uncertain environments, where model errors and stochastic disturbances are present, the GP-MPC framework exhibits resilience, adapting to uncertainty and achieving satisfactory control performance despite initial challenges in the control horizon. The exploration of reference tracking capabilities within the GP-MPC framework for the Van der Pol oscillator emphasizes its importance in attaining accurate control and ensuring compliance with predefined target setpoints. By adjusting control inputs to regulate system behavior towards target setpoints, GP-MPC demonstrates the ability to facilitate transitions from points outside stable limit cycles to stabilizable points, both inside and near the stable limit cycle. Through simulation results, we observe that while GP-MPC achieves faster settling times for displacement and velocity towards the reference points, it encounters initial challenges in controlling the system during the early stages of the control horizon, necessitating more control effort compared to direct MPC.

In our approach, the incorporation of model uncertainty into both modeling and planning stands out as a pivotal aspect, particularly in navigating complex environments. By acknowledging and addressing the inherent uncertainty in system dynamics, our method facilitates targeted exploration, allowing the system to gather essential information and refine its understanding over time. Moreover, this consideration enables us to approach constraints in a risk-averse manner, crucially important, especially in the early stages of learning, where overly optimistic actions could lead to undesired outcomes. By embracing model uncertainty, our approach not only enhances adaptability and robustness but also fosters safer and more informed decision-making in practical applications.

The key to our GP-MPC algorithm lies in redefining the optimal control problem to incorporate uncertainty propagation through moment matching, thereby transforming it into a deterministic optimal control problem. By integrating Model Predictive Control (MPC), our approach enables immediate updates to the learned model, enhancing robustness against model inaccuracies. Through empirical validation via simulation results, we have demonstrated that our framework not only excels as an efficient controller but also exhibits remarkable data efficiency, particularly in learning environments characterized by uncertainties. This achievement places our approach at the forefront of data-efficient control methodologies.


\subsection{Future Research}
Building upon the insights gained from this project work, future research endeavors can explore several promising avenues to further enhance the capabilities and applicability of the proposed GP-MPC framework. One potential direction for advancement lies in optimizing the long-term prediction model to enhance its speed and accuracy. By leveraging advanced modeling techniques and algorithmic improvements, researchers can work towards developing a more efficient and reliable long-term prediction model, capable of providing faster and more accurate predictions in real-time scenarios.

Additionally, exploring the application of GP-MPC in Multi-Input Multi-Output (MIMO) systems opens up new avenues for research. By adapting the framework to handle multiple inputs and outputs simultaneously, researchers can address complex control challenges in interconnected systems, paving the way for advanced control strategies in diverse domains ranging from robotics to industrial automation.

Furthermore, extending the application of the GP-MPC framework to real-world systems both linear and nonlinear systems, presents an exciting opportunity for future investigation. Conducting experiments and validations on physical systems can provide invaluable insights into the practical feasibility and performance of the proposed approach in real-time control applications. 


% This work offers an introduction on how to write a successful bachelor or matser's thesis at the Institute of Control Systems.

% Surely, not all your questions have been answered. Ask your supervisor and use the opportunity that he corrects a chapter before handing in your final thesis, to improve your scientific writing!