\section{Introduction}
\label{sec:motivation}
\begin{frame}
    \frametitle{Motivation}
    \begin{itemize}
        \item Learning Non-linear systems
        \item System identification requires Persistent Excitation condition %the input signal must be sufficiently rich to excite all dynamic modes of the system
        \item Drawbacks of data-driven approaches
	   \begin{itemize}
            \item learning process is slow 
		\item requires an large number of interactions
            \item Data inefficiency makes control learning of robotic systems and other complex systems impractical  %Data inefficiency makes learning in control and robotic systems impractical and obstructs the use of data-driven methods in more complex situations.
	   \end{itemize}
        % \item Requires different data-driven approaches for different type of system 
        % \item Model-based methods are susceptible to model errors
        \item State and input constraints 

    \end{itemize}
\end{frame}	

\begin{frame}     \frametitle{Approach}
\begin{itemize}
    \item Model Predictive Control(MPC)     % \item Task-specific prior knowledge 
    % \item The extraction of more information
    \item Model-Based Learning
    \item Gaussian process
    \item A transition model is proposed for long-term prediction  

\end{itemize}
\centering
\begin{tikzpicture}[node distance=2cm, auto, thick, >=triangle 45]

    % Define block styles
    \tikzstyle{block} = [rectangle, draw, fill=blue!20, 
        text width=5em, text centered, rounded corners, minimum height=2em]
    \tikzstyle{line} = [draw, -latex']
    \tikzstyle{cloud} = [draw, ellipse,fill=red!20, minimum height=2em]

    % Define nodes
    \node [cloud] (plant) {Plant};
    % \node [block, right of=plant, node distance=4cm] (data) {Data};
    \node [block, above of=plant, node distance=2cm] (gpmodel) {GP Model};
    \node [block, right of=gpmodel, node distance=6cm] (transition) {Transition Model};
    % \node [block, below of=posterior, node distance=3cm] (transition) {Transition Model};
    \node [block, below of=transition, node distance=2cm] (mpc) {MPC};

    % Draw edges
    \path [line] (plant) -- node {Data} (gpmodel);
    % \path [line] (data) -- (gpmodel);
    \path [line] (gpmodel) -- node {Posterior Prediction} (transition);
    % \path [line] (posterior) -- (transition);
    \path [line] (transition) -- node {Predictions}(mpc);
    \path [line] (mpc) -- node {optimal $u^*(1)$}  (plant);
    
\end{tikzpicture}
\end{frame}

% \begin{frame}     \frametitle{Approach}


% % % 
% %  \begin{figure}[ht]		% h - here, t - top, b - bottom, p - page, ! - try hard
% %   \centering
% %   \afig{0.7}{figures/introduction/blockdia}			% {scaling}{Figure from MATLAB, picture, etc.}
% %   \caption{Block Diagram of GP-MPC Controller Framework}
% %   \label{f:figure1}
% % \end{figure}

% \end{frame}