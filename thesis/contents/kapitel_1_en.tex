\section{Writing with \LaTeX}\label{a:latex_en}

\subsection{\LaTeX's Features}

This section gives some useful hints to write a thesis with \LaTeX. It is important to know that \LaTeX\ is not a WYSIWYG (what you see is what you get) program like other text editors, such as Microsoft Word.
Instead it much more resembles a programming language, in which you construct your text by proper usage of syntax. The ''source code'' is your \LaTeX file (\texttt{.tex}). 

As in other programming languages it is possible to insert comments in \LaTeX\ that are not visible in the final text. 
Line wraps are not of importance while writing the text, since they are created during compilation. Therefore, formatting 
with \LaTeX is not a big deal and should not take a lot of time, if all the logical relations are correct.

One of the big advantages of \LaTeX\ is writing formulas. Using the logical referencing of your formulas, those references are 
always correct, even if you change the position of the formulas. Furthermore the print quality you achieve with \LaTeX\ formulas 
is hardly matched by any other program, let alone free of charge!
Citing is very easy in \LaTeX, as well.


The following text is not very meaningful on its own, but if you read the source code at the same time, it is easy to understand how 
different elements are constructed. You should try to compile the file yourself and compare the results. If there are any differences,
check if your \LaTeX-configurations are correct.


\subsection{Continuous Text}

Writing continuous text is very easy. You can use an arbitrary text editor and just start writing, without paying any attention to 
formatting, line breaks, etc. If you want to start a new paragraph, just leave one or more blank lines...

So here we are now in a new paragraph. The font size is defined in the header of your main file. Emphasis is possible by different font styles. 
Thus when you define a new term, this is normally accentuated in \textit{italic}, important statements in \textbf{bold} and programming code in
\texttt{typewriter} or \verb"verbatim" style.

\subsubsection{Structuring Your Work}

On the lowest level you can use the command \texttt{subsubsection} to structure your work.

\subsubsection*{Same Level, but Without Numbering}

Have a look at the source code to see how that is achieved!

\subsection{Special Objects}


\subsubsection{Figures}\label{b:picturesubsection}

Figure~\ref{f:picfirstfigure} shows a block diagram.

\begin{figure}[ht]		% h - here, t - top, b - bottom, p - page, ! - try hard
  \centering
  \afig{1}{figures/example}			% {scaling}{Figure from MATLAB, picture, etc.}
  \caption{First figure}
  \label{f:picfirstfigure}
\end{figure}

If you compile your file using \texttt{dvi} 
(\texttt{latex} $\rightarrow$ \texttt{dvi} $\rightarrow$ \texttt{pdf}, oder \texttt{latex} $\rightarrow$ \texttt{dvi} $\rightarrow$ \texttt{ps} $\rightarrow$ \texttt{pdf}) 
your Figure~\ref{f:picfirstfigure} can be either a PS or an EPS file. If you are compiling with \texttt{pdfTeX}
(\texttt{latex} $\rightarrow$ \texttt{pdf}), the figures need to be stored as JPEG, PDF, PNG, etc.\ files---not in PS or EPS format.

Figure~\ref{f:xfigfigure} is an example for a diagram, created with XFig, a free and open source vector graphics editor.
Other notable vector graphics editors are
\begin{itemize}
	\item Inkscape,
	\item \LaTeX Draw,
	\item ...
\end{itemize}

\begin{figure}[ht]
  \centering
  %\xfig{1}{carts.fig}			% {scaling}{XFig figure}
  \caption{The second figure}
  \label{f:xfigfigure}
\end{figure}

\subsubsection{Formulas}

Formulas like
\begin{eqnarray}
\ddot{\phi}_1
    &=&
    \frac{M_1+l_1 \sin \phi_2
    (m_2 l_{s2} + m_3 l_{s3})
    (\dot{\phi}_2^2+2 \dot{\phi}_1\dot{\phi}_2 )
    -f_1 \dot{\phi}_1}
    {\theta_1+\theta_2+\theta_3+2l_1 (m_1 l_{s2}+m_3 l_{s3} \cos \phi_2)+
    m_3 (l_1^2+l_2^2)+m_l l_1^2}  \,,  \label{e:eqnfirst} \\
\ddot{\phi}_2
    &=&
    \frac{M_2 + l_1 \sin \phi_2 (m_2 l_{s2} + m_3 l_{s3})
    \dot{\phi}_1^2 +2 - \phi_2 \dot{\phi}_2}
    {\theta_1+\theta_2+\theta_3} \, \label{e:eqnsecond}
\end{eqnarray}
are typeset nicely. Keep in mind that formulas are and should be written as part of sentences and thus can and have to contain punctuation marks!


\subsubsection{Symbols}

Symbols like $\Omega$ can be included inline with the text. \textbf{Never} start a sentence with a symbol!


\subsubsection{Citations}

To cite a reference use the command \verb'\cite{We14}' as done here: \cite{We14}. Here \texttt{We14} is the so called BiB\TeX-Key. Have a look into the file ``S:/Standards/BibTex/rts''
to see what that means. References usually are simply attached at the end of a statement, separated by a comma,
\cite{We14}. Be aware that \LaTeX\ in general provides the command \verb'\cite'.

\subsubsection{Index of Contents}

The table of contents is generated automatically by \LaTeX. Therefore, each time you compile your main file, an \texttt{aux}-file (auxiliary) is generated that contains all 
information for the table of contents. This file is embedded, when you compile a second time. This holds true for all references and any change you make with respect to these
requires to compile twice. An intuitive and simplified explanation is that \LaTeX\ just ''reads'' your files from top to bottom, taking notes in the \texttt{aux}-file while doing so.
When reading your files again, the notes tell \LaTeX\ that it came across some figure, table or equation before, just like you when studying a subject.


\subsubsection{References}\label{b:subsecrefer}

It is possible to refer to figures, equations, etc. like this: Figure~\ref{f:picfirstfigure}, Equation~(\ref{e:eqnsecond}),
Sections~\ref{b:picturesubsection}. See how easy that is. This is one of the main advantages of \LaTeX!
Note that ''Equation~(\ref{e:eqnsecond})'', e.g., is capitalized. It is regarded as the figures own name. Would you
write your own name in lower case?

While we are at it, have a look at the figure's caption. It uses capitalization like any other sentence in this document.
Why? Because it is \textbf{not} a title, but a description.

\subsubsection{Tables}

Writing tables is probably the most cumbersome writing stuff in \LaTeX\ can get---and that is saying quite something!
Have a close look at Table~\ref{t:table}. It does not use vertical rules and different line weights in respective places.
This is the way to design tables. If you feel the need to use vertical rules, your table is probably not very readable in
the first place.

\begin{table}
	\caption{A beautiful table}
	\label{t:table}
	\begin{center}
	\begin{tabular}{ccccc}
	\toprule
                          	&        & \multicolumn{3}{c}{Columns} \\
    \cmidrule{3-5}
                            & Items  & A     & B     & C           \\
    \midrule
	\multirow{3}{*}{Rows}   & 1      & A1    & B1    & C1          \\
	                        & 2      & A2    & B2    & C2          \\
	                        & 3      & A3    & B3    & C3          \\
	\bottomrule
	\end{tabular}
	\end{center}
\end{table}

Inform yourself about some basic rules on typesetting tables:

\href{http://tug.org/pracjourn/2007-1/mori/mori.pdf}{``\textit{Tables in \LaTeX\ 2${}_\varepsilon$: Packages and Methods}''}.
